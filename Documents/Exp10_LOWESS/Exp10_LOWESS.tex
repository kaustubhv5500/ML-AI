
% Default to the notebook output style

    


% Inherit from the specified cell style.




    
\documentclass[11pt]{article}

    
    
    \usepackage[T1]{fontenc}
    % Nicer default font (+ math font) than Computer Modern for most use cases
    \usepackage{mathpazo}

    % Basic figure setup, for now with no caption control since it's done
    % automatically by Pandoc (which extracts ![](path) syntax from Markdown).
    \usepackage{graphicx}
    % We will generate all images so they have a width \maxwidth. This means
    % that they will get their normal width if they fit onto the page, but
    % are scaled down if they would overflow the margins.
    \makeatletter
    \def\maxwidth{\ifdim\Gin@nat@width>\linewidth\linewidth
    \else\Gin@nat@width\fi}
    \makeatother
    \let\Oldincludegraphics\includegraphics
    % Set max figure width to be 80% of text width, for now hardcoded.
    \renewcommand{\includegraphics}[1]{\Oldincludegraphics[width=.8\maxwidth]{#1}}
    % Ensure that by default, figures have no caption (until we provide a
    % proper Figure object with a Caption API and a way to capture that
    % in the conversion process - todo).
    \usepackage{caption}
    \DeclareCaptionLabelFormat{nolabel}{}
    \captionsetup{labelformat=nolabel}

    \usepackage{adjustbox} % Used to constrain images to a maximum size 
    \usepackage{xcolor} % Allow colors to be defined
    \usepackage{enumerate} % Needed for markdown enumerations to work
    \usepackage{geometry} % Used to adjust the document margins
    \usepackage{amsmath} % Equations
    \usepackage{amssymb} % Equations
    \usepackage{textcomp} % defines textquotesingle
    % Hack from http://tex.stackexchange.com/a/47451/13684:
    \AtBeginDocument{%
        \def\PYZsq{\textquotesingle}% Upright quotes in Pygmentized code
    }
    \usepackage{upquote} % Upright quotes for verbatim code
    \usepackage{eurosym} % defines \euro
    \usepackage[mathletters]{ucs} % Extended unicode (utf-8) support
    \usepackage[utf8x]{inputenc} % Allow utf-8 characters in the tex document
    \usepackage{fancyvrb} % verbatim replacement that allows latex
    \usepackage{grffile} % extends the file name processing of package graphics 
                         % to support a larger range 
    % The hyperref package gives us a pdf with properly built
    % internal navigation ('pdf bookmarks' for the table of contents,
    % internal cross-reference links, web links for URLs, etc.)
    \usepackage{hyperref}
    \usepackage{longtable} % longtable support required by pandoc >1.10
    \usepackage{booktabs}  % table support for pandoc > 1.12.2
    \usepackage[inline]{enumitem} % IRkernel/repr support (it uses the enumerate* environment)
    \usepackage[normalem]{ulem} % ulem is needed to support strikethroughs (\sout)
                                % normalem makes italics be italics, not underlines
    \usepackage{mathrsfs}
    

    
    
    % Colors for the hyperref package
    \definecolor{urlcolor}{rgb}{0,.145,.698}
    \definecolor{linkcolor}{rgb}{.71,0.21,0.01}
    \definecolor{citecolor}{rgb}{.12,.54,.11}

    % ANSI colors
    \definecolor{ansi-black}{HTML}{3E424D}
    \definecolor{ansi-black-intense}{HTML}{282C36}
    \definecolor{ansi-red}{HTML}{E75C58}
    \definecolor{ansi-red-intense}{HTML}{B22B31}
    \definecolor{ansi-green}{HTML}{00A250}
    \definecolor{ansi-green-intense}{HTML}{007427}
    \definecolor{ansi-yellow}{HTML}{DDB62B}
    \definecolor{ansi-yellow-intense}{HTML}{B27D12}
    \definecolor{ansi-blue}{HTML}{208FFB}
    \definecolor{ansi-blue-intense}{HTML}{0065CA}
    \definecolor{ansi-magenta}{HTML}{D160C4}
    \definecolor{ansi-magenta-intense}{HTML}{A03196}
    \definecolor{ansi-cyan}{HTML}{60C6C8}
    \definecolor{ansi-cyan-intense}{HTML}{258F8F}
    \definecolor{ansi-white}{HTML}{C5C1B4}
    \definecolor{ansi-white-intense}{HTML}{A1A6B2}
    \definecolor{ansi-default-inverse-fg}{HTML}{FFFFFF}
    \definecolor{ansi-default-inverse-bg}{HTML}{000000}

    % commands and environments needed by pandoc snippets
    % extracted from the output of `pandoc -s`
    \providecommand{\tightlist}{%
      \setlength{\itemsep}{0pt}\setlength{\parskip}{0pt}}
    \DefineVerbatimEnvironment{Highlighting}{Verbatim}{commandchars=\\\{\}}
    % Add ',fontsize=\small' for more characters per line
    \newenvironment{Shaded}{}{}
    \newcommand{\KeywordTok}[1]{\textcolor[rgb]{0.00,0.44,0.13}{\textbf{{#1}}}}
    \newcommand{\DataTypeTok}[1]{\textcolor[rgb]{0.56,0.13,0.00}{{#1}}}
    \newcommand{\DecValTok}[1]{\textcolor[rgb]{0.25,0.63,0.44}{{#1}}}
    \newcommand{\BaseNTok}[1]{\textcolor[rgb]{0.25,0.63,0.44}{{#1}}}
    \newcommand{\FloatTok}[1]{\textcolor[rgb]{0.25,0.63,0.44}{{#1}}}
    \newcommand{\CharTok}[1]{\textcolor[rgb]{0.25,0.44,0.63}{{#1}}}
    \newcommand{\StringTok}[1]{\textcolor[rgb]{0.25,0.44,0.63}{{#1}}}
    \newcommand{\CommentTok}[1]{\textcolor[rgb]{0.38,0.63,0.69}{\textit{{#1}}}}
    \newcommand{\OtherTok}[1]{\textcolor[rgb]{0.00,0.44,0.13}{{#1}}}
    \newcommand{\AlertTok}[1]{\textcolor[rgb]{1.00,0.00,0.00}{\textbf{{#1}}}}
    \newcommand{\FunctionTok}[1]{\textcolor[rgb]{0.02,0.16,0.49}{{#1}}}
    \newcommand{\RegionMarkerTok}[1]{{#1}}
    \newcommand{\ErrorTok}[1]{\textcolor[rgb]{1.00,0.00,0.00}{\textbf{{#1}}}}
    \newcommand{\NormalTok}[1]{{#1}}
    
    % Additional commands for more recent versions of Pandoc
    \newcommand{\ConstantTok}[1]{\textcolor[rgb]{0.53,0.00,0.00}{{#1}}}
    \newcommand{\SpecialCharTok}[1]{\textcolor[rgb]{0.25,0.44,0.63}{{#1}}}
    \newcommand{\VerbatimStringTok}[1]{\textcolor[rgb]{0.25,0.44,0.63}{{#1}}}
    \newcommand{\SpecialStringTok}[1]{\textcolor[rgb]{0.73,0.40,0.53}{{#1}}}
    \newcommand{\ImportTok}[1]{{#1}}
    \newcommand{\DocumentationTok}[1]{\textcolor[rgb]{0.73,0.13,0.13}{\textit{{#1}}}}
    \newcommand{\AnnotationTok}[1]{\textcolor[rgb]{0.38,0.63,0.69}{\textbf{\textit{{#1}}}}}
    \newcommand{\CommentVarTok}[1]{\textcolor[rgb]{0.38,0.63,0.69}{\textbf{\textit{{#1}}}}}
    \newcommand{\VariableTok}[1]{\textcolor[rgb]{0.10,0.09,0.49}{{#1}}}
    \newcommand{\ControlFlowTok}[1]{\textcolor[rgb]{0.00,0.44,0.13}{\textbf{{#1}}}}
    \newcommand{\OperatorTok}[1]{\textcolor[rgb]{0.40,0.40,0.40}{{#1}}}
    \newcommand{\BuiltInTok}[1]{{#1}}
    \newcommand{\ExtensionTok}[1]{{#1}}
    \newcommand{\PreprocessorTok}[1]{\textcolor[rgb]{0.74,0.48,0.00}{{#1}}}
    \newcommand{\AttributeTok}[1]{\textcolor[rgb]{0.49,0.56,0.16}{{#1}}}
    \newcommand{\InformationTok}[1]{\textcolor[rgb]{0.38,0.63,0.69}{\textbf{\textit{{#1}}}}}
    \newcommand{\WarningTok}[1]{\textcolor[rgb]{0.38,0.63,0.69}{\textbf{\textit{{#1}}}}}
    
    
    % Define a nice break command that doesn't care if a line doesn't already
    % exist.
    \def\br{\hspace*{\fill} \\* }
    % Math Jax compatibility definitions
    \def\gt{>}
    \def\lt{<}
    \let\Oldtex\TeX
    \let\Oldlatex\LaTeX
    \renewcommand{\TeX}{\textrm{\Oldtex}}
    \renewcommand{\LaTeX}{\textrm{\Oldlatex}}
    % Document parameters
    % Document title
    \title{Exp10\_LOWESS}
    
    
    
    
    

    % Pygments definitions
    
\makeatletter
\def\PY@reset{\let\PY@it=\relax \let\PY@bf=\relax%
    \let\PY@ul=\relax \let\PY@tc=\relax%
    \let\PY@bc=\relax \let\PY@ff=\relax}
\def\PY@tok#1{\csname PY@tok@#1\endcsname}
\def\PY@toks#1+{\ifx\relax#1\empty\else%
    \PY@tok{#1}\expandafter\PY@toks\fi}
\def\PY@do#1{\PY@bc{\PY@tc{\PY@ul{%
    \PY@it{\PY@bf{\PY@ff{#1}}}}}}}
\def\PY#1#2{\PY@reset\PY@toks#1+\relax+\PY@do{#2}}

\expandafter\def\csname PY@tok@w\endcsname{\def\PY@tc##1{\textcolor[rgb]{0.73,0.73,0.73}{##1}}}
\expandafter\def\csname PY@tok@c\endcsname{\let\PY@it=\textit\def\PY@tc##1{\textcolor[rgb]{0.25,0.50,0.50}{##1}}}
\expandafter\def\csname PY@tok@cp\endcsname{\def\PY@tc##1{\textcolor[rgb]{0.74,0.48,0.00}{##1}}}
\expandafter\def\csname PY@tok@k\endcsname{\let\PY@bf=\textbf\def\PY@tc##1{\textcolor[rgb]{0.00,0.50,0.00}{##1}}}
\expandafter\def\csname PY@tok@kp\endcsname{\def\PY@tc##1{\textcolor[rgb]{0.00,0.50,0.00}{##1}}}
\expandafter\def\csname PY@tok@kt\endcsname{\def\PY@tc##1{\textcolor[rgb]{0.69,0.00,0.25}{##1}}}
\expandafter\def\csname PY@tok@o\endcsname{\def\PY@tc##1{\textcolor[rgb]{0.40,0.40,0.40}{##1}}}
\expandafter\def\csname PY@tok@ow\endcsname{\let\PY@bf=\textbf\def\PY@tc##1{\textcolor[rgb]{0.67,0.13,1.00}{##1}}}
\expandafter\def\csname PY@tok@nb\endcsname{\def\PY@tc##1{\textcolor[rgb]{0.00,0.50,0.00}{##1}}}
\expandafter\def\csname PY@tok@nf\endcsname{\def\PY@tc##1{\textcolor[rgb]{0.00,0.00,1.00}{##1}}}
\expandafter\def\csname PY@tok@nc\endcsname{\let\PY@bf=\textbf\def\PY@tc##1{\textcolor[rgb]{0.00,0.00,1.00}{##1}}}
\expandafter\def\csname PY@tok@nn\endcsname{\let\PY@bf=\textbf\def\PY@tc##1{\textcolor[rgb]{0.00,0.00,1.00}{##1}}}
\expandafter\def\csname PY@tok@ne\endcsname{\let\PY@bf=\textbf\def\PY@tc##1{\textcolor[rgb]{0.82,0.25,0.23}{##1}}}
\expandafter\def\csname PY@tok@nv\endcsname{\def\PY@tc##1{\textcolor[rgb]{0.10,0.09,0.49}{##1}}}
\expandafter\def\csname PY@tok@no\endcsname{\def\PY@tc##1{\textcolor[rgb]{0.53,0.00,0.00}{##1}}}
\expandafter\def\csname PY@tok@nl\endcsname{\def\PY@tc##1{\textcolor[rgb]{0.63,0.63,0.00}{##1}}}
\expandafter\def\csname PY@tok@ni\endcsname{\let\PY@bf=\textbf\def\PY@tc##1{\textcolor[rgb]{0.60,0.60,0.60}{##1}}}
\expandafter\def\csname PY@tok@na\endcsname{\def\PY@tc##1{\textcolor[rgb]{0.49,0.56,0.16}{##1}}}
\expandafter\def\csname PY@tok@nt\endcsname{\let\PY@bf=\textbf\def\PY@tc##1{\textcolor[rgb]{0.00,0.50,0.00}{##1}}}
\expandafter\def\csname PY@tok@nd\endcsname{\def\PY@tc##1{\textcolor[rgb]{0.67,0.13,1.00}{##1}}}
\expandafter\def\csname PY@tok@s\endcsname{\def\PY@tc##1{\textcolor[rgb]{0.73,0.13,0.13}{##1}}}
\expandafter\def\csname PY@tok@sd\endcsname{\let\PY@it=\textit\def\PY@tc##1{\textcolor[rgb]{0.73,0.13,0.13}{##1}}}
\expandafter\def\csname PY@tok@si\endcsname{\let\PY@bf=\textbf\def\PY@tc##1{\textcolor[rgb]{0.73,0.40,0.53}{##1}}}
\expandafter\def\csname PY@tok@se\endcsname{\let\PY@bf=\textbf\def\PY@tc##1{\textcolor[rgb]{0.73,0.40,0.13}{##1}}}
\expandafter\def\csname PY@tok@sr\endcsname{\def\PY@tc##1{\textcolor[rgb]{0.73,0.40,0.53}{##1}}}
\expandafter\def\csname PY@tok@ss\endcsname{\def\PY@tc##1{\textcolor[rgb]{0.10,0.09,0.49}{##1}}}
\expandafter\def\csname PY@tok@sx\endcsname{\def\PY@tc##1{\textcolor[rgb]{0.00,0.50,0.00}{##1}}}
\expandafter\def\csname PY@tok@m\endcsname{\def\PY@tc##1{\textcolor[rgb]{0.40,0.40,0.40}{##1}}}
\expandafter\def\csname PY@tok@gh\endcsname{\let\PY@bf=\textbf\def\PY@tc##1{\textcolor[rgb]{0.00,0.00,0.50}{##1}}}
\expandafter\def\csname PY@tok@gu\endcsname{\let\PY@bf=\textbf\def\PY@tc##1{\textcolor[rgb]{0.50,0.00,0.50}{##1}}}
\expandafter\def\csname PY@tok@gd\endcsname{\def\PY@tc##1{\textcolor[rgb]{0.63,0.00,0.00}{##1}}}
\expandafter\def\csname PY@tok@gi\endcsname{\def\PY@tc##1{\textcolor[rgb]{0.00,0.63,0.00}{##1}}}
\expandafter\def\csname PY@tok@gr\endcsname{\def\PY@tc##1{\textcolor[rgb]{1.00,0.00,0.00}{##1}}}
\expandafter\def\csname PY@tok@ge\endcsname{\let\PY@it=\textit}
\expandafter\def\csname PY@tok@gs\endcsname{\let\PY@bf=\textbf}
\expandafter\def\csname PY@tok@gp\endcsname{\let\PY@bf=\textbf\def\PY@tc##1{\textcolor[rgb]{0.00,0.00,0.50}{##1}}}
\expandafter\def\csname PY@tok@go\endcsname{\def\PY@tc##1{\textcolor[rgb]{0.53,0.53,0.53}{##1}}}
\expandafter\def\csname PY@tok@gt\endcsname{\def\PY@tc##1{\textcolor[rgb]{0.00,0.27,0.87}{##1}}}
\expandafter\def\csname PY@tok@err\endcsname{\def\PY@bc##1{\setlength{\fboxsep}{0pt}\fcolorbox[rgb]{1.00,0.00,0.00}{1,1,1}{\strut ##1}}}
\expandafter\def\csname PY@tok@kc\endcsname{\let\PY@bf=\textbf\def\PY@tc##1{\textcolor[rgb]{0.00,0.50,0.00}{##1}}}
\expandafter\def\csname PY@tok@kd\endcsname{\let\PY@bf=\textbf\def\PY@tc##1{\textcolor[rgb]{0.00,0.50,0.00}{##1}}}
\expandafter\def\csname PY@tok@kn\endcsname{\let\PY@bf=\textbf\def\PY@tc##1{\textcolor[rgb]{0.00,0.50,0.00}{##1}}}
\expandafter\def\csname PY@tok@kr\endcsname{\let\PY@bf=\textbf\def\PY@tc##1{\textcolor[rgb]{0.00,0.50,0.00}{##1}}}
\expandafter\def\csname PY@tok@bp\endcsname{\def\PY@tc##1{\textcolor[rgb]{0.00,0.50,0.00}{##1}}}
\expandafter\def\csname PY@tok@fm\endcsname{\def\PY@tc##1{\textcolor[rgb]{0.00,0.00,1.00}{##1}}}
\expandafter\def\csname PY@tok@vc\endcsname{\def\PY@tc##1{\textcolor[rgb]{0.10,0.09,0.49}{##1}}}
\expandafter\def\csname PY@tok@vg\endcsname{\def\PY@tc##1{\textcolor[rgb]{0.10,0.09,0.49}{##1}}}
\expandafter\def\csname PY@tok@vi\endcsname{\def\PY@tc##1{\textcolor[rgb]{0.10,0.09,0.49}{##1}}}
\expandafter\def\csname PY@tok@vm\endcsname{\def\PY@tc##1{\textcolor[rgb]{0.10,0.09,0.49}{##1}}}
\expandafter\def\csname PY@tok@sa\endcsname{\def\PY@tc##1{\textcolor[rgb]{0.73,0.13,0.13}{##1}}}
\expandafter\def\csname PY@tok@sb\endcsname{\def\PY@tc##1{\textcolor[rgb]{0.73,0.13,0.13}{##1}}}
\expandafter\def\csname PY@tok@sc\endcsname{\def\PY@tc##1{\textcolor[rgb]{0.73,0.13,0.13}{##1}}}
\expandafter\def\csname PY@tok@dl\endcsname{\def\PY@tc##1{\textcolor[rgb]{0.73,0.13,0.13}{##1}}}
\expandafter\def\csname PY@tok@s2\endcsname{\def\PY@tc##1{\textcolor[rgb]{0.73,0.13,0.13}{##1}}}
\expandafter\def\csname PY@tok@sh\endcsname{\def\PY@tc##1{\textcolor[rgb]{0.73,0.13,0.13}{##1}}}
\expandafter\def\csname PY@tok@s1\endcsname{\def\PY@tc##1{\textcolor[rgb]{0.73,0.13,0.13}{##1}}}
\expandafter\def\csname PY@tok@mb\endcsname{\def\PY@tc##1{\textcolor[rgb]{0.40,0.40,0.40}{##1}}}
\expandafter\def\csname PY@tok@mf\endcsname{\def\PY@tc##1{\textcolor[rgb]{0.40,0.40,0.40}{##1}}}
\expandafter\def\csname PY@tok@mh\endcsname{\def\PY@tc##1{\textcolor[rgb]{0.40,0.40,0.40}{##1}}}
\expandafter\def\csname PY@tok@mi\endcsname{\def\PY@tc##1{\textcolor[rgb]{0.40,0.40,0.40}{##1}}}
\expandafter\def\csname PY@tok@il\endcsname{\def\PY@tc##1{\textcolor[rgb]{0.40,0.40,0.40}{##1}}}
\expandafter\def\csname PY@tok@mo\endcsname{\def\PY@tc##1{\textcolor[rgb]{0.40,0.40,0.40}{##1}}}
\expandafter\def\csname PY@tok@ch\endcsname{\let\PY@it=\textit\def\PY@tc##1{\textcolor[rgb]{0.25,0.50,0.50}{##1}}}
\expandafter\def\csname PY@tok@cm\endcsname{\let\PY@it=\textit\def\PY@tc##1{\textcolor[rgb]{0.25,0.50,0.50}{##1}}}
\expandafter\def\csname PY@tok@cpf\endcsname{\let\PY@it=\textit\def\PY@tc##1{\textcolor[rgb]{0.25,0.50,0.50}{##1}}}
\expandafter\def\csname PY@tok@c1\endcsname{\let\PY@it=\textit\def\PY@tc##1{\textcolor[rgb]{0.25,0.50,0.50}{##1}}}
\expandafter\def\csname PY@tok@cs\endcsname{\let\PY@it=\textit\def\PY@tc##1{\textcolor[rgb]{0.25,0.50,0.50}{##1}}}

\def\PYZbs{\char`\\}
\def\PYZus{\char`\_}
\def\PYZob{\char`\{}
\def\PYZcb{\char`\}}
\def\PYZca{\char`\^}
\def\PYZam{\char`\&}
\def\PYZlt{\char`\<}
\def\PYZgt{\char`\>}
\def\PYZsh{\char`\#}
\def\PYZpc{\char`\%}
\def\PYZdl{\char`\$}
\def\PYZhy{\char`\-}
\def\PYZsq{\char`\'}
\def\PYZdq{\char`\"}
\def\PYZti{\char`\~}
% for compatibility with earlier versions
\def\PYZat{@}
\def\PYZlb{[}
\def\PYZrb{]}
\makeatother


    % Exact colors from NB
    \definecolor{incolor}{rgb}{0.0, 0.0, 0.5}
    \definecolor{outcolor}{rgb}{0.545, 0.0, 0.0}



    
    % Prevent overflowing lines due to hard-to-break entities
    \sloppy 
    % Setup hyperref package
    \hypersetup{
      breaklinks=true,  % so long urls are correctly broken across lines
      colorlinks=true,
      urlcolor=urlcolor,
      linkcolor=linkcolor,
      citecolor=citecolor,
      }
    % Slightly bigger margins than the latex defaults
    
    \geometry{verbose,tmargin=1in,bmargin=1in,lmargin=1in,rmargin=1in}
    
    

    \begin{document}
    
    
    \maketitle
\section{Problem Statement}
\subsection{Objective}
To implement the non-parametric Locally Weighted Regression algorithm in order to fit data points to the given data set.
\subsection{Outcomes}
\begin{enumerate}
    \item Implement the LOWESS (locally weighted scatterplot smoothing) algorithm using R.
    \item Train a locally weighted regression model with an appropriate dataset.
    \item Plot and interpret the output graphs.
\end{enumerate}
\subsection{System Requirements}Linux OS with R and relevant libraries installed
\subsection{Dataset Description}
The dataset used to train the model is formed using a sine wave with white gaussian noise added to it. The noise adds randomness to the data which helps in testing of the algorithm. \newline
No. of attributes: 2 \newline
No. of instances: 200
\subsection{Algorithm}
Locally Weighted Linear regression is a supervised learning algorithm used for computing linear relationships between input (X) and output (Y).
\begin{enumerate}
    \item Training phase: Compute $\theta$ to minimize the cost; 
    $J(\theta) = \sum_{i=1}^{m}w^i(\theta^Tx^i-y^i)^2$.
    \item Predict Output; return $\theta^Tx$.
\end{enumerate}
\section{Code and Output}
    
    \begin{Verbatim}[commandchars=\\\{\}]
{\color{incolor}In [{\color{incolor}1}]:} \PY{c+c1}{\PYZsh{} R version}
        \PY{n}{version}
\end{Verbatim}

    
    \begin{verbatim}
               _                           
platform       x86_64-pc-linux-gnu         
arch           x86_64                      
os             linux-gnu                   
system         x86_64, linux-gnu           
status                                     
major          4                           
minor          1.1                         
year           2021                        
month          08                          
day            10                          
svn rev        80725                       
language       R                           
version.string R version 4.1.1 (2021-08-10)
nickname       Kick Things                 
    \end{verbatim}

    
    \begin{Verbatim}[commandchars=\\\{\}]
{\color{incolor}In [{\color{incolor}2}]:} \PY{c+c1}{\PYZsh{} Importing the dataset from csv file}
        \PY{n}{data} \PY{o}{\PYZlt{}}\PY{o}{\PYZhy{}} \PY{n}{read}\PY{o}{.}\PY{n}{csv}\PY{p}{(}\PY{l+s+s1}{\PYZsq{}}\PY{l+s+s1}{LOWESS\PYZus{}scatter.csv}\PY{l+s+s1}{\PYZsq{}}\PY{p}{)}\PY{p}{;}
        \PY{n}{head}\PY{p}{(}\PY{n}{data}\PY{p}{)}\PY{p}{;}
        \PY{n}{X} \PY{o}{=} \PY{n}{data}\PY{p}{[}\PY{p}{,}\PY{l+m+mi}{1}\PY{p}{]}\PY{p}{;}
        \PY{n}{Y} \PY{o}{=} \PY{n}{data}\PY{p}{[}\PY{p}{,}\PY{l+m+mi}{2}\PY{p}{]}\PY{p}{;}
\end{Verbatim}

    A data.frame: 6 × 2
\begin{tabular}{r|ll}
  & X & Y\\
  & <dbl> & <dbl>\\
\hline
	1 & 0.147 & -0.2780\\
	2 & 0.152 &  0.4470\\
	3 & 0.177 &  0.0864\\
	4 & 0.204 &  0.6530\\
	5 & 0.247 &  0.5520\\
	6 & 0.251 &  0.4270\\
\end{tabular}


    
    \begin{Verbatim}[commandchars=\\\{\}]
{\color{incolor}In [{\color{incolor}3}]:} \PY{c+c1}{\PYZsh{} Helper function to implement clipping of vector}
        \PY{n}{clip} \PY{o}{\PYZlt{}}\PY{o}{\PYZhy{}} \PY{n}{function}\PY{p}{(}\PY{n}{x}\PY{p}{,} \PY{n}{lower}\PY{p}{,} \PY{n}{upper}\PY{p}{)} \PY{p}{\PYZob{}}
          \PY{n}{pmax}\PY{p}{(}\PY{n}{pmin}\PY{p}{(}\PY{n}{x}\PY{p}{,} \PY{n}{upper}\PY{p}{)}\PY{p}{,} \PY{n}{lower}\PY{p}{)}
        \PY{p}{\PYZcb{}}
        
        \PY{c+c1}{\PYZsh{} Function to implement LOWESS Algorithm}
        \PY{n}{lowess} \PY{o}{\PYZlt{}}\PY{o}{\PYZhy{}} \PY{n}{function}\PY{p}{(}\PY{n}{x}\PY{p}{,} \PY{n}{y}\PY{p}{,} \PY{n}{f} \PY{o}{=} \PY{l+m+mf}{2.}\PY{o}{/}\PY{l+m+mf}{3.}\PY{p}{,}\PY{n+nb}{iter} \PY{o}{=} \PY{l+m+mi}{3}\PY{p}{)}\PY{p}{\PYZob{}}
          \PY{n}{n} \PY{o}{=} \PY{n}{length}\PY{p}{(}\PY{n}{x}\PY{p}{)}\PY{p}{;}
          \PY{n}{r} \PY{o}{=} \PY{n}{ceiling}\PY{p}{(}\PY{n}{f}\PY{o}{*}\PY{n}{n}\PY{p}{)}\PY{p}{;}
          \PY{n}{h} \PY{o}{=} \PY{n}{c}\PY{p}{(}\PY{p}{)}\PY{p}{;}
          \PY{k}{for}\PY{p}{(}\PY{n}{i} \PY{o+ow}{in} \PY{l+m+mi}{1}\PY{p}{:}\PY{n}{n}\PY{p}{)}\PY{p}{\PYZob{}}
            \PY{n}{temp} \PY{o}{=} \PY{n+nb}{abs}\PY{p}{(}\PY{n}{x} \PY{o}{\PYZhy{}} \PY{n}{x}\PY{p}{[}\PY{n}{i}\PY{p}{]}\PY{p}{)}\PY{p}{;}
            \PY{n}{sort}\PY{p}{(}\PY{n}{temp}\PY{p}{)}
            \PY{n}{h}\PY{p}{[}\PY{n}{i}\PY{p}{]} \PY{o}{=} \PY{n}{temp}\PY{p}{[}\PY{n}{r}\PY{p}{]}\PY{p}{;}
          \PY{p}{\PYZcb{}}
          \PY{n}{minV} \PY{o}{\PYZlt{}}\PY{o}{\PYZhy{}} \PY{l+m+mf}{0.0}\PY{p}{;}
          \PY{n}{maxV} \PY{o}{\PYZlt{}}\PY{o}{\PYZhy{}} \PY{l+m+mf}{1.0}\PY{p}{;}
          \PY{n}{x1} \PY{o}{=} \PY{n}{matrix}\PY{p}{(}\PY{n}{x}\PY{p}{,}\PY{l+m+mi}{1}\PY{p}{,}\PY{n}{n}\PY{p}{)}\PY{p}{;}
          \PY{n}{x2} \PY{o}{=} \PY{n}{matrix}\PY{p}{(}\PY{n}{x}\PY{p}{,}\PY{n}{n}\PY{p}{,}\PY{l+m+mi}{1}\PY{p}{)}\PY{p}{;}
          \PY{n}{w} \PY{o}{=} \PY{n}{matrix}\PY{p}{(}\PY{l+m+mi}{0}\PY{p}{,}\PY{l+m+mi}{1}\PY{p}{,}\PY{n}{n}\PY{p}{)}\PY{p}{;}
          \PY{k}{for}\PY{p}{(}\PY{n}{i} \PY{o+ow}{in} \PY{l+m+mi}{1}\PY{p}{:}\PY{n}{n}\PY{p}{)}\PY{p}{\PYZob{}}
            \PY{n}{temp} \PY{o}{=} \PY{n}{matrix}\PY{p}{(}\PY{n+nb}{abs}\PY{p}{(}\PY{p}{(}\PY{n}{x1}\PY{p}{[}\PY{n}{i}\PY{p}{]} \PY{o}{\PYZhy{}} \PY{n}{x2}\PY{p}{)}\PY{o}{/}\PY{n}{h}\PY{p}{[}\PY{n}{i}\PY{p}{]}\PY{p}{)}\PY{p}{,}\PY{l+m+mi}{1}\PY{p}{,}\PY{n}{n}\PY{p}{)}\PY{p}{;}
            \PY{n}{temp} \PY{o}{=} \PY{n}{clip}\PY{p}{(}\PY{n}{temp}\PY{p}{,}\PY{n}{minV}\PY{p}{,}\PY{n}{maxV}\PY{p}{)}\PY{p}{;}
            \PY{n}{w} \PY{o}{=} \PY{n}{rbind}\PY{p}{(}\PY{n}{w}\PY{p}{,}\PY{n}{temp}\PY{p}{)}\PY{p}{;}
          \PY{p}{\PYZcb{}}
          \PY{n}{w} \PY{o}{=} \PY{n}{w}\PY{p}{[}\PY{l+m+mi}{1}\PY{p}{:}\PY{n}{n}\PY{p}{,}\PY{p}{]}\PY{p}{;}
          \PY{n}{w} \PY{o}{=} \PY{p}{(}\PY{l+m+mi}{1} \PY{o}{\PYZhy{}} \PY{n}{w} \PY{o}{*}\PY{o}{*} \PY{l+m+mi}{3}\PY{p}{)} \PY{o}{*}\PY{o}{*} \PY{l+m+mi}{3}\PY{p}{;}
          \PY{n}{yest} \PY{o}{=} \PY{n}{matrix}\PY{p}{(}\PY{l+m+mi}{0}\PY{p}{,}\PY{l+m+mi}{1}\PY{p}{,}\PY{n}{n}\PY{p}{)}\PY{p}{;}
          \PY{n}{delta} \PY{o}{=} \PY{n}{matrix}\PY{p}{(}\PY{l+m+mi}{1}\PY{p}{,}\PY{l+m+mi}{1}\PY{p}{,}\PY{n}{n}\PY{p}{)}\PY{p}{;}
          \PY{n}{prev\PYZus{}weights} \PY{o}{=} \PY{n}{matrix}\PY{p}{(}\PY{l+m+mi}{0}\PY{p}{,}\PY{l+m+mi}{1}\PY{p}{,}\PY{n}{n}\PY{p}{)}\PY{p}{;}
        
          \PY{k}{for}\PY{p}{(}\PY{n}{iteration} \PY{o+ow}{in} \PY{l+m+mi}{1}\PY{p}{:}\PY{n+nb}{iter}\PY{p}{)}\PY{p}{\PYZob{}}
            \PY{k}{for}\PY{p}{(}\PY{n}{i} \PY{o+ow}{in} \PY{l+m+mi}{1}\PY{p}{:}\PY{n}{n}\PY{p}{)}\PY{p}{\PYZob{}}
              \PY{n}{weights} \PY{o}{=} \PY{n}{delta} \PY{o}{*} \PY{n}{matrix}\PY{p}{(}\PY{n}{w}\PY{p}{[}\PY{p}{,}\PY{n}{i}\PY{p}{]}\PY{p}{,}\PY{l+m+mi}{1}\PY{p}{,}\PY{n}{n}\PY{p}{)}\PY{p}{;}
              \PY{k}{if}\PY{p}{(}\PY{n+nb}{any}\PY{p}{(}\PY{o+ow}{is}\PY{o}{.}\PY{n}{na}\PY{p}{(}\PY{n}{weights}\PY{p}{)}\PY{p}{)}\PY{p}{)}\PY{p}{\PYZob{}}
                \PY{n}{weights} \PY{o}{=} \PY{n}{prev\PYZus{}weights}\PY{p}{;}
              \PY{p}{\PYZcb{}}
              \PY{n}{b} \PY{o}{=} \PY{n}{c}\PY{p}{(}\PY{n+nb}{sum}\PY{p}{(}\PY{n}{weights}\PY{o}{*}\PY{n}{y}\PY{p}{)}\PY{p}{,} \PY{n+nb}{sum}\PY{p}{(}\PY{n}{weights}\PY{o}{*}\PY{n}{y}\PY{o}{*}\PY{n}{x}\PY{p}{)}\PY{p}{)}\PY{p}{;}
              \PY{n}{A} \PY{o}{=} \PY{n}{matrix}\PY{p}{(}\PY{n}{c}\PY{p}{(}\PY{n+nb}{sum}\PY{p}{(}\PY{n}{weights}\PY{p}{)}\PY{p}{,} \PY{n+nb}{sum}\PY{p}{(}\PY{n}{weights}\PY{o}{*}\PY{n}{x}\PY{p}{)}\PY{p}{,}
              \hspace{0.6cm} \PY{n+nb}{sum}\PY{p}{(}\PY{n}{weights}\PY{o}{*}\PY{n}{x}\PY{p}{)}\PY{p}{,}\PY{n+nb}{sum}\PY{p}{(}\PY{n}{weights}\PY{o}{*}\PY{n}{x}\PY{o}{*}\PY{n}{x}\PY{p}{)}\PY{p}{)}\PY{p}{,}\PY{l+m+mi}{2}\PY{p}{,}\PY{l+m+mi}{2}\PY{p}{)}\PY{p}{;}
              \PY{n}{beta} \PY{o}{=} \PY{n}{solve}\PY{p}{(}\PY{n}{A}\PY{p}{,}\PY{n}{b}\PY{p}{)}\PY{p}{;}
              \PY{n}{yest}\PY{p}{[}\PY{n}{i}\PY{p}{]} \PY{o}{=} \PY{n}{beta}\PY{p}{[}\PY{l+m+mi}{1}\PY{p}{]} \PY{o}{+} \PY{n}{beta}\PY{p}{[}\PY{l+m+mi}{2}\PY{p}{]} \PY{o}{*} \PY{n}{x}\PY{p}{[}\PY{n}{i}\PY{p}{]}\PY{p}{;}
              \PY{n}{prev\PYZus{}weights} \PY{o}{=} \PY{n}{weights}\PY{p}{;}
            \PY{p}{\PYZcb{}}
            
            \PY{c+c1}{\PYZsh{} Calculating the residual and median of residuals}
            \PY{n}{residuals} \PY{o}{=} \PY{n}{matrix}\PY{p}{(}\PY{n}{y}\PY{p}{,}\PY{l+m+mi}{1}\PY{p}{,}\PY{n}{n}\PY{p}{)} \PY{o}{\PYZhy{}} \PY{n}{yest}\PY{p}{;}
            \PY{n}{residuals} \PY{o}{=} \PY{n}{matrix}\PY{p}{(}\PY{n}{residuals}\PY{p}{,}\PY{l+m+mi}{1}\PY{p}{,}\PY{n}{n}\PY{p}{)}\PY{p}{;}
        
            \PY{n}{s} \PY{o}{=} \PY{n}{sort}\PY{p}{(}\PY{n+nb}{abs}\PY{p}{(}\PY{n}{residuals}\PY{p}{)}\PY{p}{)}\PY{p}{[}\PY{n}{n}\PY{o}{/}\PY{l+m+mi}{2}\PY{p}{]}\PY{p}{;}
            \PY{n}{delta} \PY{o}{=} \PY{n}{clip}\PY{p}{(}\PY{n}{residuals} \PY{o}{/} \PY{p}{(}\PY{l+m+mf}{6.0} \PY{o}{*} \PY{n}{s}\PY{p}{)}\PY{p}{,} \PY{o}{\PYZhy{}}\PY{l+m+mi}{1}\PY{p}{,} \PY{l+m+mi}{1}\PY{p}{)}\PY{p}{;}
            \PY{n}{delta} \PY{o}{=} \PY{p}{(}\PY{l+m+mi}{1} \PY{o}{\PYZhy{}} \PY{n}{delta} \PY{o}{*}\PY{o}{*} \PY{l+m+mi}{2}\PY{p}{)} \PY{o}{*}\PY{o}{*} \PY{l+m+mi}{2}\PY{p}{;}
            \PY{n}{delta} \PY{o}{=} \PY{n}{matrix}\PY{p}{(}\PY{n}{delta}\PY{p}{,}\PY{l+m+mi}{1}\PY{p}{,}\PY{n}{n}\PY{p}{)}\PY{p}{;}
          \PY{p}{\PYZcb{}}
          \PY{k}{return}\PY{p}{(}\PY{n}{yest}\PY{p}{)}\PY{p}{;}
        \PY{p}{\PYZcb{}}
\end{Verbatim}

    \begin{Verbatim}[commandchars=\\\{\}]
{\color{incolor}In [{\color{incolor}4}]:} \PY{n}{yest} \PY{o}{=} \PY{n}{lowess}\PY{p}{(}\PY{n}{X}\PY{p}{,}\PY{n}{Y}\PY{p}{,}\PY{n}{f}\PY{o}{=}\PY{l+m+mf}{0.25}\PY{p}{,}\PY{n+nb}{iter}\PY{o}{=}\PY{l+m+mi}{15}\PY{p}{)}\PY{p}{;}
        \PY{n}{library}\PY{p}{(}\PY{n}{ggplot2}\PY{p}{)}\PY{p}{;}
        \PY{n}{plot}\PY{p}{(}\PY{n}{X}\PY{p}{,}\PY{n}{Y}\PY{p}{,}\PY{n+nb}{type}\PY{o}{=}\PY{l+s+s1}{\PYZsq{}}\PY{l+s+s1}{l}\PY{l+s+s1}{\PYZsq{}}\PY{p}{,}\PY{n}{col}\PY{o}{=}\PY{l+s+s1}{\PYZsq{}}\PY{l+s+s1}{red}\PY{l+s+s1}{\PYZsq{}}\PY{p}{,}\PY{n}{lwd}\PY{o}{=}\PY{l+m+mi}{2}\PY{p}{,}\PY{n}{main}\PY{o}{=}\PY{l+s+s2}{\PYZdq{}}\PY{l+s+s2}{Output Graph}\PY{l+s+s2}{\PYZdq{}}\PY{p}{)}\PY{p}{;}
        \PY{n}{lines}\PY{p}{(}\PY{n}{X}\PY{p}{,}\PY{n}{yest}\PY{p}{,}\PY{n+nb}{type}\PY{o}{=}\PY{l+s+s1}{\PYZsq{}}\PY{l+s+s1}{l}\PY{l+s+s1}{\PYZsq{}}\PY{p}{,}\PY{n}{col}\PY{o}{=}\PY{l+s+s1}{\PYZsq{}}\PY{l+s+s1}{green}\PY{l+s+s1}{\PYZsq{}}\PY{p}{,}\PY{n}{lwd}\PY{o}{=}\PY{l+m+mi}{2}\PY{p}{)}\PY{p}{;}
        \PY{n}{legend}\PY{p}{(}\PY{n}{x}\PY{o}{=}\PY{l+m+mf}{8.45}\PY{p}{,}\PY{n}{y}\PY{o}{=}\PY{l+m+mi}{3}\PY{p}{,}\PY{n}{legend}\PY{o}{=}\PY{n}{c}\PY{p}{(}\PY{l+s+s1}{\PYZsq{}}\PY{l+s+s1}{Y with Noise}\PY{l+s+s1}{\PYZsq{}}\PY{p}{,}\PY{l+s+s1}{\PYZsq{}}\PY{l+s+s1}{Locally Weighted Regression Output}\PY{l+s+s1}{\PYZsq{}}\PY{p}{)}\PY{p}{,}
        \PY{n}{col}\PY{o}{=}\PY{n}{c}\PY{p}{(}\PY{l+s+s2}{\PYZdq{}}\PY{l+s+s2}{red}\PY{l+s+s2}{\PYZdq{}}\PY{p}{,}\PY{l+s+s2}{\PYZdq{}}\PY{l+s+s2}{green}\PY{l+s+s2}{\PYZdq{}}\PY{p}{)}\PY{p}{,}\PY{n}{cex}\PY{o}{=}\PY{l+m+mf}{0.6}\PY{p}{,}\PY{n}{pch}\PY{o}{=}\PY{l+m+mi}{15}\PY{p}{)}\PY{p}{;}
\end{Verbatim}

    \begin{center}
    \adjustimage{max size={0.8\linewidth}{0.9\paperheight}}{lowess.png}
    \end{center}

    
\section{Interpretation and Conclusion}
\subsection{Interpretation of Output}
From the output, we can interpret that:
\begin{itemize}
    \item Based on the training data, the LOWESS Algorithm tries to generalize and smooth the fitted curve by minimizing the cost function and using the locally calculated weights.
    \item For the given sinusoidal data with noise, we can see that the algorithm effectively fits and predicts the output for new data points while smoothing the output curve.
\end{itemize}

\subsection{Conclusion}
From the experiment conducted, it can be concluded that:
\begin{itemize}
    \item Local regression is a generalization of moving average and polynomial regression. Its most common method is the LOWESS (locally weighted scatterplot smoothing) which is a non-parametric regression method that combines multiple regression models in a k-nearest-neighbor-based meta-model.
    \item The algorithm uses weights and cost functions to calculate and minimize errors while fitting the dataset.
    \item However, LOWESS makes less efficient use of data than other least squares methods. It requires fairly large, densely sampled data sets in order to produce good models. This is because LOWESS relies on the local data structure when performing the local fitting. Thus, LOWESS provides less complex data analysis in exchange for greater experimental costs.
\end{itemize}
    % Add a bibliography block to the postdoc
    
    
    
    \end{document}
